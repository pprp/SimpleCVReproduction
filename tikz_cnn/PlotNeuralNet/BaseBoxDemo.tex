\documentclass[border=8pt, multi, tikz]{standalone} 
\usetikzlibrary{quotes,arrows.meta,positioning,3d}
\usepackage{xcolor}
\definecolor{salmon}{RGB}{250, 128, 114} % custom color
\usepackage{layer/Box}
\begin{document}
\begin{tikzpicture}

% define the size of the box
\xdef\x{5};
\xdef\y{5};
\xdef\z{5};

% get the box
\pic[shift={(0,0,0)}] at (0,0,0) 
    {BaseBox={
        name=demo,
        caption=\textbf{caption} as well as \textcolor{green}{f a c e},
        opacity=0.6,
        xlabel=x axis, % text appears on x axis
        ylabel=y axis, % text appears on y axis
        zlabel=z axis, % text appears on z axis
        fill=salmon, % I have defined the color in line 4
        scriptscale=1.1, % upscale the size of text
        width=\x,
        height=\y,
        depth=\z,
        border=black
        }
    };

% an example to demonstrate the relative position and how beautiful the result is.
\pic[shift={(3,0,0)}] at (demo-nearright) 
    {BaseBox={
        name=blank,
        caption=solid material,
        opacity=1,
        xlabel=, % text appears on x axis
        ylabel=, % text appears on y axis
        zlabel=, % text appears on z axis
        fill=salmon, % mixture of color, I hate this mothod
        scriptscale=1.1, % upscale the size of text
        width=\x,
        height=\y,
        depth=\z,
        border=none
        }
    };

% show the nodes. One can refer to the nodes when drawing
\node at (0 , \y/2 , \z/2) {a}; 
\node at (0 ,-\y/2 , \z/2) {b}; 
\node at (\x    ,-\y/2 , \z/2) {c}; 
\node at (\x    , \y/2 , \z/2) {d}; 
\node at (\x    , \y/2 ,-\z/2) {e}; 
\node at (\x    ,-\y/2 ,-\z/2) {f}; 
\node at (0 ,-\y/2 ,-\z/2) {g}; 
\node at (0 , \y/2 ,-\z/2) {h};
\node [color=green] at (demo-back) {b a c k};
\node [color=red, xslant=1.1, yscale=1.15] at (demo-head) {h e a d};
\node [color=red, xslant=1.1, yscale=1.15] at (demo-bottom) {b o t t o m};
\node [color=blue, yslant=1.2, yscale=1.1] at (demo-left) {l e f t};
\node [color=blue, yslant=1.2, yscale=1.1] at (demo-right) {r i g h t};
\node [color=blue, yslant=1.2, yscale=1.1] at (demo-nearleft) {n e a r   l e f t};
\node [color=blue, yslant=1.2, yscale=1.1] at (demo-nearright) {n e a r    r i g h t};
\draw [opacity=0.2](demo-left) -- ++(-2, 0, 0);
\draw [opacity=0.2](demo-right) -- ++(2,0,0);

\end{tikzpicture}
\end{document}
